\documentclass[conference]{IEEEtran}

% correct bad hyphenation here
\hyphenation{op-tical net-works semi-conduc-tor}

\usepackage{graphicx}

% on table
\usepackage{array}

% on equasion
\usepackage[only,varodot]{stmaryrd}
\usepackage[euler]{textgreek}
\usepackage{MnSymbol}

\graphicspath{{./img/}}


\begin{document}

\title{Review On Recently Various Full Adder Designs}

\author{
	\IEEEauthorblockN{Jun Huang}
	\IEEEauthorblockA{
		Gina Cody School of Engineering and Computer Science\\
		Concordia University\\
		Email: youyinnn@foxmail.com
	}
}
\maketitle

\begin{abstract}
	The abstract goes here.
\end{abstract}

\begin{IEEEkeywords}
	Full Adder, 1-bit Adder, Hybrid Adder, CMOS,
\end{IEEEkeywords}

\section{Introduction}

...

A 22T XOR-XNOR-based hybrid full adder design \cite{20212210429416} has been proposed in 2021
which uses an XOR-XNOR module combined with the sum-generation module which uses TG technique and the carry-generation module which use CCMOS technique.
% As the designers discussed, it is a scalable and full-swing FA
% with some performance improvements compared with several existing state-of-the-art FAs.

A 18T hybrid full adder design \cite{9339799} has been  proposed in 2020.
This design is also a three modules cell based on a XOR-XNOR circuit.
With this fundamental circuit, two 4T circuits, which use the GDI technique and TG technique respectively, are driven to generate sum and carry.

In section \ref{sec:b}, the 22T XOR-XNOR-based hybrid FA is discussed as FA-1;
Then in section \ref{sec:c}, the 18T hybrid FA is discussed as FA-2;

% 列举其他找到的设计 但是并不在报告里讨论 因为设计的论文本身提供的信息在不做模拟的情况下难以放在同一个标准下进行比较 或者论文本身缺失了参数信息

\section{FA-1: XOR-XNOR-based Hybrid FA}
\label{sec:b}

\subsection{Design}

The FA-1 \cite{20212210429416} design contains three modules to perform a 1-bit full adder operation.
Firstly, an XOR-XNOR circuit will take original inputs \(A\) and \(B\) as its input and produce two signals,
one is from the XNOR gate marked as \(S_{xnor\_out}\) and the other is from the XOR gate marked as \( S_{xor\_out}\).
Then, a TG-based circuit as the second module will take both signal and the input carry \(C_{in}\) to calculate the \(Sum\)
while a third module will then use the \(C_{in}\) along with \(A\), \(B\) and signals \(S_{xnor\_out}\), \( S_{xor\_out}\) to generate the carry \(C_{out}\).

Fig. \ref{fig:fa1-bd} presents the block diagram of the design.

\begin{figure}[!ht]
	\centering
	\includegraphics[width=2.7in]{fa1-block diagram.png}
	\caption{Block Diagram of FA-1}
	\label{fig:fa1-bd}
\end{figure}

\subsubsection{Module 1}XOR-XNOR Circuit

The circuit consists of 10 transistors as is shown in Fig. \ref{fig:fa1-xor-xnor}.

\begin{figure}[!ht]
	\centering
	\includegraphics[width=2.2in]{fa1-xor-xnor circuit.png}
	\caption{XOR-XNOR Circuit of FA-1}
	\label{fig:fa1-xor-xnor}
\end{figure}

For all the possible \(A\), \(B\) inputs,
table \ref{tb:xor-xnor} presents all the output patterns also the responsible transistors within the circuit.
As the table presented, it is clear that at least one transistor path can provide full swing output to prevent threshold voltage loss.

\begin{table*}[!htb]
	\renewcommand{\arraystretch}{1.3}
	\caption{Operation Table of the XOR-XNOR circuit of FA-1}
	\centering
	\begin{tabular}{p{1.5cm}p{1.5cm}p{1.5cm}p{4cm}p{4cm}l}
		\hline
		\multicolumn{6}{ c }{\bfseries XOR Circuit}                                                                                                                                                                                     \\
		\multicolumn{3}{ c }{\bfseries Input} & \multicolumn{2}{ c }{\bfseries Output Transistor Path} & \multicolumn{1}{ c }{\bfseries Output}                                                                                         \\
		Pattern no.                           & \(A\)                                                  & \(B\)                                  & Full Swing Transistor Path & Non-Full Swing Transistor Path & Output Signal\(/\)logic \\
		\hline
		1.                                    & 0                                                      & 0                                      & \(n_3\)                    & \(p_1/p_2\)                    & \(B/0\)                 \\
		2.                                    & 0                                                      & 1                                      & \(p_2\)                    & \(n_1/n_3\)                    & \(B/1\)                 \\
		3.                                    & 1                                                      & 0                                      & \(p_1\)                    & None                           & \(A/1\)                 \\
		4.                                    & 1                                                      & 0                                      & \(n_1\) and \(n_2\)        & None                           & \(0\)                   \\
		\hline
		\multicolumn{6}{ c }{\bfseries XNOR Circuit}                                                                                                                                                                                    \\
		\multicolumn{3}{ c }{\bfseries Input} & \multicolumn{2}{ c }{\bfseries Output Transistor Path} & \multicolumn{1}{ c }{\bfseries Output}                                                                                         \\
		Pattern no.                           & \(A\)                                                  & \(B\)                                  & Full Swing Transistor Path & Non-Full Swing Transistor Path & Output Signal\(/\)logic \\
		\hline
		1.                                    & 0                                                      & 0                                      & \(p_3\) and \(p_4\)        & None                           & \(1\)                   \\
		2.                                    & 0                                                      & 1                                      & \(n_5\)                    & None                           & \(A/0\)                 \\
		3.                                    & 1                                                      & 0                                      & \(n_4\)                    & \(p_4/p_5\)                    & \(B/0\)                 \\
		4.                                    & 1                                                      & 0                                      & \(p_5\)                    & \(n_4/n_5\)                    & \(B/1\)                 \\
		\hline
	\end{tabular}
	\label{tb:xor-xnor}
\end{table*}

\subsubsection{Module 2}Sum Generation Circuit

From the Table \ref{tb:fa-tf}, two paths for the Sum generation can be concluded:

when \(C_{in} = 0\):
\[
	Sum = A \oplus B
\]

when \(C_{in} = 1\):
\[
	Sum = A \varodot B
\]

\begin{table}[!ht]
	\renewcommand{\arraystretch}{1.3}
	\caption{True-False Table of FA}
	\centering
	\begin{tabular}{p{1cm}p{1cm}p{1cm}p{1cm}p{1cm}}
		\hline
		\multicolumn{3}{ c }{\bfseries Inputs} & \multicolumn{2}{ c }{\bfseries Outputs}                                 \\
		\(C_{in}\)                             & \(A\)                                   & \(B\) & \(C_{out}\) & \(Sum\) \\
		\hline
		0                                      & 0                                       & 0     & 0           & 0       \\
		                                       & 0                                       & 1     & 0           & 1       \\
		                                       & 1                                       & 0     & 0           & 1       \\
		                                       & 1                                       & 1     & 1           & 0       \\
		1                                      & 0                                       & 0     & 0           & 1       \\
		                                       & 0                                       & 1     & 1           & 0       \\
		                                       & 1                                       & 0     & 1           & 0       \\
		                                       & 1                                       & 1     & 1           & 1       \\
		\hline
	\end{tabular}
	\label{tb:fa-tf}
\end{table}

For input selection, a 2:1 Multiplexer(2:1 MUX) based on Transmission Gate(TG) has been used for figuring the \(Sum\).

As is given in Fig. \ref{fig:fa1-circuit}, two TGs are used for implementing 2:1 MUX logic and with this manner,
the Sum generation circuit can also provide full swing output.

\begin{figure}[!ht]
	\centering
	\includegraphics[width=2.7in]{fa1-circuit.png}
	\caption{FA-1 Circuit}
	\label{fig:fa1-circuit}
\end{figure}

\subsubsection{Module 3}Carry Generation Circuit

As per Fig. \ref{fig:fa1-circuit}, the Carry generation circuit uses a CCMOS logic-based inverter at the output.
With this logic, the output voltage level of this circuit is either \(V_{dd}\) or \(G_{nd}\) level.
It is a good design for the extensibility of the circuit
since the design of the carry generation logic is the most important factor of the scalability of the wide word length adders.


\subsection{Discussion}

FA-1 shows impressive improvements compared to many other FAs.
The presenters are intended to solve the voltage degradation issue with the hybrid logic-based design,
hence they design the XOR-XNOR-based circuit to produce full swing output for its next stage
and choose TG for implementing the sum generation circuit.

Finally, for the purpose of high-speed calculations and good scalabilities,
they apply the CCMOS technique at the output terminial.
The designed carry generation circuit can be extended to multiple bits adder structures with no additional voltage level restoring buffers.
Furthermore, this logic can reduce the carry chain delay with involving just one pull-up and pull-down transistors.

As for the area overhead, a total of 22 transistors are used to produce this FA, namely, 10 on module one, 4 on module 2, and 8 on module 3.

According to the presenters' simulation using 45nm technique and 1 voltage supply on 1-bit FA,
compare with the conventional CMOS FA \cite{weste2015cmos} and other recent FA designs \cite{9068497,18743001},
% this design has 19.35\% improvement in Silicon area, 33.59\% improvement in Average Power, 36.15\% improvement in Propagation Delay, 56.22\% improvement in Area Delay Product, and 57.59\% improvement in Power Delay Product.
this design achieves many improvements which can be shown in the Table \ref{tb:fa1-comparison}.

\begin{table*}[!ht]
	\renewcommand{\arraystretch}{1.3}
	\caption{Part Of The 1-bit FA Performance Investigation Performed By The Designers of FA-1}
	\centering
	\begin{tabular}{l c l l l l l *{5}{p{1cm}}}
		\hline
		\multicolumn{1}{c}{\bfseries FA Cell } & \bfseries TC & \multicolumn{5}{ c }{\bfseries Performance} & \multicolumn{5}{ c }{\bfseries Improvements with Respect to CCMOS FA}                                                                                                                                                                  \\
		                                       &              & Area(\textmugreek\(m^2\))                   & AP(\textmugreek\(W\))                                                 & PD(ps)         & ADP(\textmugreek\(m^2\).ps) & PDP(\textalpha J) & Area($\slashdiv$) & AP($\slashdiv$) & PD($\slashdiv$) & ADP($\slashdiv$) & PDP($\slashdiv$) \\
		\hline
		\cite{weste2015cmos} CCMOS             & 28           & 10.13                                       & 1.28                                                                  & 60.3           & 610.84                      & 77.18             & 0.00              & 0.00            & 0.00            & 0.00             & 0.00             \\
		\cite{9068497} Kandpal's               & 20           & 9.18                                        & 0.92                                                                  & 54.06          & 496.27                      & 49.74             & 9.38              & 28.13           & 10.35           & 18.76            & 35.55            \\
		\cite{18743001} Sanapala's             & 14           & \bfseries 7.38                              & 0.75                                                                  & 56.7           & 415.45                      & 42.53             & \bfseries 27.15   & \bfseries 41.41 & 5.97            & 31.99            & 44.90            \\
		\cite{20212210429416} FA-1             & 22           & 8.17                                        & 0.85                                                                  & \bfseries 38.5 & \bfseries 267.40            & \bfseries 32.73   & 19.35             & 33.59           & \bfseries 36.15 & \bfseries 56.22  & \bfseries 57.59  \\
		\hline
	\end{tabular}
	\label{tb:fa1-comparison}
\end{table*}

Again, based on the presenters' simulations of the word-length adder architecture \cite{20212210429416}, the FA-1 continues to rank first in these performance comparisons with its scalability advantages, , especially on 16-bit and 32-bit.

\section{FA-2: 18T Hybrid FA}
\label{sec:c}

\subsection{Design}

The FA-2 \cite{9339799} design contains three module. Same as FA-1 \cite{20212210429416},
the module 1 of the FA-2 is an XOR-XNOR circuit that sending out two signals to drive Module 2 and Module 3 for the calculation of \(Sum\) and \(Carry\).

Module 1 from FA-2 first take \(A\) and \(B\) and produce signal \(H\) (result of $A \oplus B$) and signal $\overline{H}$ (result of $A \varodot B$).
Module 2 then take \(H\) and \(C_{in}\) to calculate the \(Sun\) ($C_{in} \oplus H$).
Module 3 finally figure the \(C_{out}\) out from the input \(A\) and \(C_{in}\) also count in the selection between \(H\) and $\overline{H}$.

The expressions of those outcomes can be:

\begin{equation}
	Sum = H\overline{C_{in}} + \overline{H}C_{in} = A \oplus B \oplus C_{in}
\end{equation}
\begin{equation}
	C_{out} = HC_{in} + \overline{H}A
\end{equation}

\subsubsection{Module 1}XOR-XNOR Circuit

The structure of module 1 is presented in Fig. \ref{fig:fa2-m1}.
To avoid the lack of voltage level, a feedback loop is attached to the XOR output and XNOR output
for reaching the power required to drive the next two modules. So it use a pMOS transistor as Mpf, a nMOS transistor as Mnf.
Any weak logic signal produced by the XOR gate or the XNOR gate (when \(A=B\))
will be handle by the feedback loop and obtains perfect logic signal at the \(H\) or $\overline{H}$ output.
For example, if \(A = B = 0\), weak logic 0 will be produce at the XOR gate as \(H\),
but it still turns ON the Mpf and then logic 1 passes to the Mnf, thus a perfect logic 0 is taken at the \(H\).

\begin{figure}[!ht]
	\centering
	\includegraphics[width=2.8in]{fa2-m1.png}
	\caption{Module 1 of FA-2}
	\label{fig:fa2-m1}
\end{figure}

\subsubsection{Module 2} Sum Generation Circuit

The structure of module 2 is presented in Fig. \ref{fig:fa2-m2}.
Sum generate implementation of FA-2 is an XOR gate with Gate Diffusion Input technique.
So it can generate the full swing output.
When using GDI on the XOR gate, it is required that an inverter provide a perfect signal to the inputs.
But since it retrieves such signal from module 1, the implementation of this module will only need 4 transistors to finish the task.

\begin{figure}[!hb]
	\centering
	\includegraphics[width=2.8in]{fa2-m2.png}
	\caption{Module 2 of FA-2}
	\label{fig:fa2-m2}
\end{figure}

\subsubsection{Module 2} Carry Generation Circuit

The structure of module 3 is presented in Fig. \ref{fig:fa2-m3}.
The circuit is a 4 pass transistors(PT) 2-to-1 MUX, and it will provide a full swing output by a pMOS and an nMOS conducted in each of the input patterns.
For instance, when \(H = A = 0\), Mp1 and Mn2 turn on, a logic 0 goes through these transistors, \(C_{out}\) obtains perfect 0.

\begin{figure}[!ht]
	\centering
	\includegraphics[width=2.8in]{fa2-m3.png}
	\caption{Module 3 of FA-2}
	\label{fig:fa2-m3}
\end{figure}

\subsection{Discussion}

FA-2 design is a lean logic cell for the addition operation.
The designers were well aware of the problems such as bad performance happens at low supply voltages
and the stability of the produced output at different loading conditions.
The design combined feedback loop design, GDI design, and PT technique all together to solve these problems.

With respect of area overhead, 18 transistors are used to bulid this cell. 10 on module 1, 4 on module 2, and 4 on module 3.

As simulations performed by the designers which presented in Table \ref{tb:fa2-comparison},
FA-2 shows its advantages on speed and PDP aspects compared with one old hybrid CMOS FA cell and one hybrid FA.

\begin{table}[!ht]
	\renewcommand{\arraystretch}{1.3}
	\caption{Simulations Performed By The Designers of FA-2 \\ in 90{\upshape nm} 1.2V}
	\centering
	\begin{tabular}{l c *{3}{p{1.2cm}}}
		\hline
		\multicolumn{1}{c}{\bfseries FA Cell } & \bfseries TC & \multicolumn{3}{ c }{\bfseries Performance}                                      \\
		                                       &              & AP(\textmugreek\(W\))                       & PD(ps)          & PDP(fJ)          \\
		\hline
		\cite{9262027} Hybrid CCMOS            & 24           & 6.21                                        & 143             & 0.888            \\
		\cite{15484823} New Hybrid             & 16           & 1.1766                                      & 91.3            & 0.107            \\
		\cite{9339799} FA-2                    & 18           & \bfseries 1.104                             & \bfseries 85.22 & \bfseries 0.0941 \\
		\hline
	\end{tabular}
	\label{tb:fa2-comparison}
\end{table}

\section{Design C}

An XOR-XNOR-based hybrid full adder design has been proposed\cite{20212210429416}
which uses an XOR-XNOR module combined with the carry-generation module and the sum-generation module.
As the designers discussed, it is a scalable and full-swing FA
with some performance improvements compared with several existing state-of-the-art FAs.An XOR-XNOR-based hybrid full adder design has been proposed\cite{20212210429416}
which uses an XOR-XNOR module combined with the carry-generation module and the sum-generation module.
As the designers discussed, it is a scalable and full-swing FA
with some performance improvements compared with several existing state-of-the-art FAs.

\section{Comparison}

An XOR-XNOR-based hybrid full adder design has been proposed\cite{20212210429416}
which uses an XOR-XNOR module combined with the carry-generation module and the sum-generation module.
As the designers discussed, it is a scalable and full-swing FA
with some performance improvements compared with several existing state-of-the-art FAs.An XOR-XNOR-based hybrid full adder design has been proposed\cite{20212210429416}
which uses an XOR-XNOR module combined with the carry-generation module and the sum-generation module.
As the designers discussed, it is a scalable and full-swing FA
with some performance improvements compared with several existing state-of-the-art FAs.

\section{Conclusion}

An XOR-XNOR-based hybrid full adder design has been proposed\cite{20212210429416}
which uses an XOR-XNOR module combined with the carry-generation module and the sum-generation module.
As the designers discussed, it is a scalable and full-swing FA
with some performance improvements compared with several existing state-of-the-art FAs.An XOR-XNOR-based hybrid full adder design has been proposed\cite{20212210429416}
which uses an XOR-XNOR module combined with the carry-generation module and the sum-generation module.
As the designers discussed, it is a scalable and full-swing FA
with some performance improvements compared with several existing state-of-the-art FAs.

\bibliographystyle{ieeetr}
\bibliography{bibliographies}
\end{document}
