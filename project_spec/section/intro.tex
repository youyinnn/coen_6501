\section{Introduction}
\maskcite{cervantes1999}
En un lugar de la Mancha, de cuyo nombre no quiero acordarme, no ha mucho tiempo que vivía un hidalgo de los de lanza en astillero, adarga antigua, rocín flaco y galgo corredor.
\subsection{Heading II}
Una olla de algo más vaca que carnero, salpicón las más noches, duelos y quebrantos los sábados, lantejas los viernes, algún palomino de añadidura los domingos, consumían las tres partes de su hacienda.
\subsubsection{Heading III}
El resto della concluían sayo de velarte, calzas de velludo para las fiestas, con sus pantuflos de lo mesmo, y los días de entresemana se honraba con su vellorí de lo más fino.
\paragraph{Heading IV}
Tenía en su casa una ama que pasaba de los cuarenta, y una sobrina que no llegaba a los veinte, y un mozo de campo y plaza, que así ensillaba el rocín como tomaba la podadera.
Frisaba la edad de nuestro hidalgo con los cincuenta años; era de complexión recia, seco de carnes, enjuto de rostro, gran madrugador y amigo de la caza.
\subparagraph{Heading V}
Quieren decir que tenía el sobrenombre de Quijada, o Quesada, que en esto hay alguna diferencia en los autores que deste caso escriben; aunque por conjeturas verosímiles se deja entender que se llamaba Quijana.
