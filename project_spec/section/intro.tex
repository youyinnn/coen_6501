\section{Introduction}

This is the project report of the course \textbf{COEN 6501: Digital Design and Synthesis}.
The project requires an implementation of an ALU circuit design that perform the equation of \textbf{\(Z = \frac{1}{4} [A^2\ast B] + 1\)}.
More details of the requirements will be discussed in the next section.
Breaking down the project requirements, the team should implement the following component designs to complete the ALU:

\begin{table}[!ht]
	\renewcommand{\arraystretch}{1.3}
	\caption{Components of the ALU Circuit}
	\centering
	\begin{tabular}{ >{\centering\arraybackslash}p{7cm} >{\centering\arraybackslash}p{7cm} }
		\hline
		\bfseries Functional Components        & \bfseries Other Components \\
		\hline
		n-bit adder design                     & 2-bit right shifter        \\
		n-bit multiplier with 2 and 3 operands & n-bit incrementor          \\
		overflow handler                       & negative edge registers    \\
		\(END\_FLAG\) generator                &                            \\
		\hline
	\end{tabular}
	\label{tb:cac}
\end{table}

As per adder design, the team chose the \textbf{\textquote{Carry Select Adder}} implementation.
This adder design is suitable for bit width extension.

As per multiplier design, the team chose the \textbf{\textquote{Radix-4 Booth Algorithm}} implementation
which is an elegant design for reducing both area occupation and delay.

As per the overflow handler, the team provides two solutions:
1. giving a negation which is impossible for the demanded equation to indicate the overflow;
2. dividing the full result into the higher part and the lower part and outputting them to the \(Z\) port alternately within two clock cycles.

As per the \(END\_FLAG\) generator, the team chose to implement a FSM for indicating the valid \(END\_FLAG\) signal.
As per the register, falling edge clock event sensitive register will be used in the project,
the registers for input \(A\) and \(B\) should use \(LOAD\) signal as the clock event which means those two registers are asynchronous,
the rest of the registers should use \(Clock\) signal as the clock event.

After completing all component designs, the ALU will be implemented in both pipelined and non-pipelined designs.

\noindent The project is designed, implemented, simulated, and synthesized with the environments of:
\begin{itemize}
	\item \textit{\textbf{ModelSim} SE-64 6.6g May 23 2012 Linux 3.10.0-1160.45.1.el7.x86\_64}
	\item \textit{\textbf{Precision RTL Synthesis}  64-bit 2016.1.0.15 (Production Release)}
	\item \textit{\textbf{XILINX ISE and Project Navigator}  10.1 (lin64)}
	\item \textit{\textbf{GHDL} 2.0.0-dev (1.0.0.r849.gc56db233)}
\end{itemize}

\noindent The synthesized report is based on the \textquote{2VP20ff896} device with default timing constraints provided by \textit{Precision}.
