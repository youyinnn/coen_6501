\section{Synthesis and Analysis of the Arithmetic Circuit}

\subsection{Introduction}

The arithmetic circuit is synthesized for implementation on a field-programmable grid array (FPGA).
Xilinx Virtex-II Pro is used as the FPGA in this project.
Mentor Graphics PrecisionRTL is used to generate and synthesize the circuits,
which also details the area usage and can perform timing analysis if needed.
Xilinx ISE is used for getting the timing and power information.
% Also, the timing analysis is done by Xilinx ISE, the result of which is shown in Appendix X.

For timing information, two main operation modules of the arithmetic circuit are considered.
The reciprocal of the largest delay is taken to determine the maximum frequency.
The equation between delay and maximum frequency is as follows:

\begin{equation}
	\frac{1}{T_{delay}} = Frequency
	\label{exp:dl}
\end{equation}


\subsection{Timing Information}
\subsubsection{Radix-4 Booth Multiplier}
The timing analysis was performed with Xilinx ISE.
The default settings were applied since there were no timing constraints used.
The longest delay was found between mp(3) and p(14) at 15.793 ns.
By using the equation presented above, the maximum frequency is 63.32 MHz.
The full timing report from Xilinx ISE is shown in \appref{sec:ap_4}.

\subsubsection{CSA}
\paragraph{16-bit CSA}
The timing analysis was performed with Xilinx ISE.
The default settings were applied since there were no timing constraints used.
The longest delay was found between in\_a(0) and carryout at 8.490 ns.
By using the equation presented above, the maximum frequency is 117.79 MHz.
The full timing report from Xilinx ISE is shown in \appref{sec:ap_5}.

\paragraph{24-bit CSA}
The timing analysis was performed with Xilinx ISE.
The default settings were applied since there were no timing constraints used.
The longest delay was found between in\_a(0) and sum(22) at 9.675 ns.
By using the equation presented above, the maximum frequency is 103.36 MHz.
The full timing report from Xilinx ISE is shown in \appref{sec:ap_6}.

\subsubsection{Implementation Circuit}
The full timing report from Xilinx ISE for non-pipelined version and pipelined version regarding
the timing source values are shown in \appref{sec:ap_7}, \appref{sec:ap_8}, and \appref{sec:ap_9}.

\subsection{Area and Power Results for the Non-Pipelined Version}
\subsubsection{Implementation with Negation Output}
\paragraph{Area}

When synthesizing the non-pipelined version, the total number of slices used was 377 or 4.06\% 
of all available logic slices. The full report from Mentor Graphics PrecisionRTL regarding the area
and resource used is shown in \appref{sec:ap_1}.

\paragraph{Power}

\noindent The power information was performed with Xilinx ISE. The results are shown in the \tbref{tb:non_p_power}.

\begin{table}[!ht]
	\renewcommand{\arraystretch}{1.3}
	\caption{Power Information for Non-pipelined Version}
	\centering
	\begin{tabular}{ p{4cm} p{4cm} p{4cm} }
		\hline
		\bfseries Source & \bfseries Voltage(\(V\)) & \bfseries Power(\(W\)) \\
		\hline
		Vccint           & 1.5                      & 0.06                   \\
		Vccaux           & 2.5                      & 0.025                  \\
		Vcco25           & 2.5                      & 0.003                  \\
		\hline
		                 &                          & Total power: 0.088W    \\
	\end{tabular}
	\label{tb:non_p_power}
\end{table}

\subsection{Area and Power Results for the Pipelined Version}
\subsubsection{Implementation with Negation Output}
\paragraph{Area}
When synthesizing the pipelined negation output version, the total number of slices used was 380 or 4.09\% 
of all available logic slices. The full report from Mentor Graphics PrecisionRTL regarding the area
and resource used is shown in \appref{sec:ap_2}.

\paragraph{Power}
The power information was performed with Xilinx ISE. The results are shown in the \tbref{tb:p__neg_power}.

\begin{table}[!ht]
	\renewcommand{\arraystretch}{1.3}
	\caption{Power Information for Pipelined Version with Negation Output}
	\centering
	\begin{tabular}{ p{4cm} p{4cm} p{4cm} }
		\hline
		\bfseries Source & \bfseries Voltage(\(V\)) & \bfseries Power(\(W\)) \\
		\hline
		Vccint           & 1.5                      & 0.06                   \\
		Vccaux           & 2.5                      & 0.025                  \\
		Vcco25           & 2.5                      & 0.003                  \\
		\hline
		                 &                          & Total power: 0.088W    \\
	\end{tabular}
	\label{tb:p__neg_power}
\end{table}

\subsubsection{Implementation with Separation Output}

When synthesizing the pipelined separation output version, the total number of slices used was 379 or 4.08\% 
of all available logic slices. The full report from Mentor Graphics PrecisionRTL regarding the area
and resource used is shown in \appref{sec:ap_3}.

\paragraph{Power}
The power information was performed with Xilinx ISE. The results are shown in the \tbref{tb:p_sep_power}.

\begin{table}[!ht]
	\renewcommand{\arraystretch}{1.3}
	\caption{Power Information for Pipelined Version with Separation Output}
	\centering
	\begin{tabular}{ p{4cm} p{4cm} p{4cm} }
		\hline
		\bfseries Source & \bfseries Voltage(\(V\)) & \bfseries Power(\(W\)) \\
		\hline
		Vccint           & 1.5                      & 0.06                   \\
		Vccaux           & 2.5                      & 0.025                  \\
		Vcco25           & 2.5                      & 0.003                  \\
		\hline
		                 &                          & Total power: 0.088W    \\
	\end{tabular}
	\label{tb:p_sep_power}
\end{table}
